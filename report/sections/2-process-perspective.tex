\section{Process's perspective} \label{section:Process perspective}
\todo{Also reflect and describe what was the "DevOps" style of your work. For example, what did you do differently to previous development projects and how did it work?}
\subsection{Interactions and organization of developer team} %jesper

\subsection{CI/CD chain} %jesper

\subsection{Organization of repository} %jesper
The code is organized using a mono repository setup, having all code and scripts necessary to run the application gathered in a single repository.%, \underline{\href{https://github.com/DevOps2021-gb/devops2021}{devops2021}.}

\subsection{Applied branching strategy and development process} %jesper

\subsection{Monitoring} %Jonas

\subsection{Logging + experiment}
Logging is set up as an EFK stack. \textcolor{red}{What did we log initially? Just warnings and exceptions/errors?} After logging was added and experiment was conducted to see whether it was possible to use the logging to detect bugs in the system. Three different bugs were introduced and one of the could be detected in the logging. Afterwards more detailed logging was introduced now also logging requests and their payloads. This change made all thre kinds of bug show up in the log. The description of the experiment can be found in the wiki entry \underline{\href{https://github.com/DevOps2021-gb/devops2021/wiki/Catch-a-Bug-By-Looking-at-the-Logs}{Catch a Bug By Looking at the Logs}}.
\textcolor{red}{How do we aggregate logs?}

\subsection{Security assessment self and from group a/L} %Jonas

\subsection{Scaling and load balancing}\label{subsection:scaling} %Nikolaj/Frederik
