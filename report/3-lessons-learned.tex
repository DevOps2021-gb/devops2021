\section{Lessons learned perspective} \label{section:Lessons learned perspective}
\subsection{Issues: evolution and refactoring}
Initially, the inherited code was translated to a basic Java application using the Java Spark framework\footnote{\url{https://github.com/DevOps2021-gb/devops2021/issues/152}}. At the time, this code was very bare-bones with all logic and endpoints located in a single file. As time went on, the team refactored the application to make use of a more well structured system architecture (controller-, service-, persistance-layer)\footnote{\url{https://github.com/DevOps2021-gb/devops2021/tree/ae6cd48a06fde2a5c403056b33895b7589cb039d}} eventually adding dependency injection and mock testing\footnote{\url{https://github.com/DevOps2021-gb/devops2021/commit/07914442a09e204f0be2687fe8e5b9bd07799ee4}} to the application. 
During development one of the endpoint's response code was changed which led to a massive increase in errors the simulator reported, which could have been avoided if a fraction of the provided \texttt{minitwit\_simulator.py} file had been a part of the test pipeline.

\subsection{Issues: operation}
\label{issues-operation}
The self-made heartbeat protocol had some issues related to Spark. An issue\footnote{\url{https://github.com/DevOps2021-gb/devops2021/issues/171}} was created trying to remedy this but was never finished, rendering the availability part of the system unstable as the backup would perceive the primary to be down and then reassign the floating-ip to itself exposing a single point of failure. The issue is likely related to Spark's session handling. These scripts can be found in the repository\footnote{\url{https://github.com/DevOps2021-gb/devops2021/tree/main/heartbeat}}

Early on in the project, the database was deployed together with the Minitwit application using Vagrant. Due to SSH key mismanagement, we lost access to our vagrant deployed DigitalOcean Minitwit server containing simulator generated data. Luckily, we were able to reset the root user through Digital Ocean's web terminal functionality. Through this terminal we could add SSH public keys to the \textit{authorized\_keys} file. Finally, we could then SSH into the server and get a database dump. This database dump was based on an outdated model with no ORM, so we had to alter the tables before transferring it to the new remote database server. Only a minor amount of data was lost during the data transfer. To minimize data loss in the future, we created a backup GitHub Action that made a database backup once a day. In addition, SSH keys for the server were backed up in a keystore.

On the 18th of March the webserver crashed due to SSL exceptions. After looking through Docker logs, we narrowed it down to an incompatibility between java-mysql-connector 8.0.15 and Java 11. The solution to this was to update java-mysql-connector 8.0.15 to 8.0.23 \footnote{\url{https://github.com/DevOps2021-gb/devops2021/issues/94}}.

\subsection{Issues: maintenance}
\label{issues-maintenance}
The team never figured out how to make SonarCloud read the test coverage\footnote{\url{https://github.com/DevOps2021-gb/devops2021/issues/140}}, eventually scrapping the idea after spending much time trying to make it work. It should, however, be said that it is able to identify which test functions are proper test functions and that code coverage is obtainable through an IDE like IntelliJ.

Furthermore, the team tried to implement alert notifications on Grafana\footnote{\url{https://github.com/DevOps2021-gb/devops2021/issues/87}} in order to be notified of unexpected data from the monitoring, but this was never successfully implemented.

As the only manual part of the release process, the team had to rename the previous release from 'development release' to the correct tag/version. An issue\footnote{\url{https://github.com/DevOps2021-gb/devops2021/issues/170}} was created in an attempt to automate the entire release process and have an internal counter of the current version, but was never solved.

\subsection{Reflection on DevOps style of work}
The automatic and continuous deployment was very nice to work with since it made it possible for all team members to deploy code without understanding some complicated set of steps to go through. In earlier projects team members have experienced cases where only few group members knew how to deploy.

The idea of having containers, which should enable all members to run the code regardless of tools and dependencies was nice, but the team experienced issues with this resulting in all members still not being able to run the application locally. The issues occurred after adding the EFK stack.

Regarding the way of working together, structuring the repository and branching, this way was not new to any group members but the technique worked well. Some of the techniques group members had used were: making flow visible, limiting work in process, reducing batch/issue sizes and the different ways of eliminating waste in the value stream. In regards to the second way, mobilizing mix of experts and people needing the knowledge has been common practice, and requiring approval near source and experts was common practice. As this is a student project the third ways of creating high-trust culture came naturally.

Adding tests to CI was effective as it removed assumptions and increased resilience of anything committed - and multiple times stopped broken code, which otherwise would have been approved. \\
The logging and maintenance were helpful to provide feedback, allowing us to see the problems as they occurred.
